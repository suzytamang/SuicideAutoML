\noindent Suicide is a devastating public and private health problem that disproportionately impacts our veterans \cite{} and active-duty military \cite{}.  Scientists from across disciplines and sectors are now actively involved in understanding the problem of suicide. In 1975 there were 576 PubMed publications that mentioned the word `suicide' and in 2016 there were 3,578.  Although this literature is directly relevant for improving models that seek to target individuals at most risk of a suicide attempt \cite{}, subject matter experts can no longer keep up with the pace of publication \cite{cohen2014biomedical}.  Similarly, computational experts face challenges meeting the demands for updating their models to reflect new inputs \cite{wijeratne_feature_2017}, especially when the predictors of interest are outside the standard types of clinical information contained within observational healthcare databases---e.g., social determinants of health or a patient's functional status \cite{gonzalez-hernandez_capturing_2017}.

Using suicide prevention as a motivating use case, our \underline{long-term goal} is to build a cross-cutting  AutoML (automated machine learning) platform that can be used to fully automate the evolution of machine learning models that integrate newly discovered knowledge about an individual's risk or protective factors.  Recent work in other fields has demonstrated the feasibility of deploying AutoML to address real-world machine learning tasks and the advantages that have been reported include the ability to produce simpler solutions, faster creation of those solutions, and models that often outperform hand-crafted models \cite{}.  Our work seeks to build on more general AutoML frameworks to design and validate an end-to-end solution for population-level risk stratification.  Our solution couples knowledge extracted from the literature in combination with the ontology-guided representation of new prediction model features and examines their impact on model behavior.  

Our \underline{central hypothesis} is that AutoML can remove some bottlenecks associated with the deployment of machine learning models; by deploying alternative systems and comparing their results, we can develop a design framework to support the evolution of human-centered machine learning systems for suicide prevention.   For example, it's no longer feasible for suicide experts to keep pace with the volume of new studies published on suicide in the scientific literature; for machine learning practitioners, tasks such as hyper-parameter optimization to maximize performance of model can be tedious, time consuming,  error prone, and frequently do not generalize well \cite{davidhand,coref,stephenwu}.  Such a framework can help the VA and other organizations to more effectively deploy suicide prediction models for alternative data and compute environments; it can also help to empower the types of interdisciplinary teams that build human-centered machine learning systems for suicide prevention and for other high priority areas.

We plan to test our central hypothesis and, thereby, accomplish our overall objective for this project by pursuing the following three specific aims: 

\textbf{Aim 1: Mine new scientific knowledge about risk and protective factors from the growing body of scientific literature, across disciplines. (John, Kevin, Mark?) KBC: personally, I would rather not do this one.  It's definitely my strongest area, but not the one where I would like to focus, if that works for everyone else.}

\noindent 1.1. Continuously survey literature to identify new suicide research papers. 
\newline 1.2. Extract new knowledge about risk and protective factors and study metadata 
\newline 1.3. Populate a suicide research knowledge-base to store the relevant study information -- i.e., `slots' for the at-risk population -- in an easily retrievable and machine-readable way

\textbf{Aim 2: Develop a framework for deriving ontology-based computable representations for suicide information extraction.(Robert, Mark, Suzanne) KBC: this one is more interesting to me.}
\newline 2.1. Customize biomedical terminologies to clinical text 
\newline 2.1.1 Use VA data to train entity embeddings and map to CUIs, focus on mental health concepts and high priority areas
\newline 2.2. Extract and represent suicide relevant information from structured and unstructured VA CDW data using VA-specific mental health concept definitions

\textbf{Aim 3: Determine the alternative machine learning algorithms, hyperparameter optimization techniques and other modeling preferences that are most relevant to suicide prediction and compare their prospective predictive performance. (Silvia, Robert, Suzanne) KBC: this is something that I'm doing a lot of right now, so it's the most interesting of the 4 aims, from my personal perspective (and everything is always all about me!) :-).}
\newline 3.1. Instantiate and evaluate linear and non-linear models
\newline 3.2. Support generative and discriminate modeling
\newline 3.3. Automate algorithm selection and hyperparameter optimization to maximize the prospective predictive performance of the final machine learning model
%\newline 3.4. Build a longitudinal model that handles events that are measured irregularly overtime as it is usually the case in health care data.
%\newline 3.5. Develop information retrieval algorithms applied to medical notes that can extract and abstract useful information.
\newline
\textbf{Aim 4: Distribute open source components to the scientific community, including language and ontology resources, and modeling code. Provide a collaborative environment for software integration, testing, and comparison. Provide an "HPC estimator" based on complexity of problem at hand.}
\newline

By comparing the strengths and limitations of various system permutations, our proposed AutoML framework is expected to identify opportunities for automating key aspects of what is now a challenging and laborious process of updating machine learning prediction models that are used in operational settings; also, the extent to which we can leverage new knowledge discoveries published in the scientific literature to generate ontology-driven computable representations for standard and more complex phenotypes (e.g., a patient's level of social connectedness).  In addition to the immediate relevance of our work to existing suicide prevention initiatives within the Department of Veterans Affairs such as REACHVET, our AutoML framework can be extended to new use cases and by other providers that wish to integrate cross-cutting analytic platforms with patient care to improve the health of the population in high-priority areas.

\bibliography{}
\bibliographystyle{natbib}
